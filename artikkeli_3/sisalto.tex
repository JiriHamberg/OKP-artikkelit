 \lstset{%
        inputencoding=utf8,
            extendedchars=true,
            literate=%
            {å}{{\r{a}}}1
			{ä}{{\"a}}1
			{ö}{{\"o}}1
			{Å}{{\r{A}}}1
			{Ä}{{\"A}}1
			{Ö}{{\"O}}1
}




\section{Nimetyt aliohjelmat ja parametrinvälitys}

\subsection{F\#}

\lstset{
	language=FSharp,
	basicstyle=\ttfamily,
	breaklines=true,
	columns=fullflexible
}

F\#:ssa funktiot määritetään lähes samalla tavalla kuin (vakio)muuttujat. Siinä missä vakiomuuttujien asetus tapahtuu seuraavasti
 
 
\begin{lstlisting} 
  let muuttujaNimi = arvo # muuttujalla siis todellakin on tyyppi, mutta kääntäjä päättelee sen kontekstista
\end{lstlisting}

funktioiden määrittely tapahtuu puolestaan

\begin{lstlisting}
  let funktio parametri1 parametri2 =
    koodia # tässä edelleen syytä muistaa että sisennys määrittelee lohkorakenteen
\end{lstlisting}

Tässä voidaan nopeasti huomata ettei kielessä onnistu parametrittoman funktion määrittely. Jos parametriluettelo on tyhjä, ei määritetäkään funktiota vaan (vakio)muuttuja. Funktioille voidaan kuitenkin määrittää Unit-tyyppinen parametri (vrt. Scalan Unit-arvo) sulkeiden avulla seuraavasti:

\begin{lstlisting}
  let parametritonFunktio () = koodi…
\end{lstlisting}

Vastaavasti funktioita kutsutaan F\#:ssa seuraavasti

\begin{lstlisting}
  funktio param1 param2  # jos kaksiparametrinen funktio.
\end{lstlisting}


Unit-tyyppinen parametri ilmoitetaan sulkeiden avulla kuten funktiomäärittelyssä:

\begin{lstlisting}
  funktio () # kutsutaan funktiota, jonka parametrityyppi on Unit
\end{lstlisting}

Funktio voidaan myös määrittää muodossa~\cite{msn_params}

\begin{lstlisting}
  let funktio(a, b, c) = koodia…
\end{lstlisting}

jolloin funktiota kutsutaan muodossa

\begin{lstlisting}
  funktio(1, 2, 3)
\end{lstlisting}

Teknisesti (1, 2, 3) on kolmikko tyyppiä int * int * int (missä * on Karteesinen tulo). Näin ollen parametrilista voi yhdistellä molempia tapoja vapaasti, esimerkiksi seuraavaan tapaan:

\begin{lstlisting}
let f (a,b,c) (d,e) = a*b*c + d*e #funktion f tyyppi on nyt int*int*int -> int*int
\end{lstlisting}

Yllä esitettyä funktiota voitaisiin kutsua vaikkapa näin: f (1,2,3) (4,5).

Monikkojen avulla määritellyt jäsenfunktiot tukevat myös nimettyjä parametreja~\cite{msn_params}, esimerkiksi

\begin{lstlisting}
  type Laskuri() =
    member this.Laske(ensimmainen, toinen) = ensimmainen - toinen

[<EntryPoint>]
let main argv =
    let laskuri = Laskuri()
    printfn "%A" (laskuri.Laske(3, 4))
    printfn "%A" (laskuri.Laske(toinen = 4, ensimmainen = 3)) # sama kuin yllä
    0
\end{lstlisting}

Kieli mahdollistaa myös valinnaiset (optional) argumentit metodeille. Nämä merkitään aloittamalla nimi kysymysmerkillä. Nämä argumentit ovat kielen Optional-tyyppiä, jolloin arvo
tai sen puuttuminen voidaan määrittää match-lausekkeen avulla. Kutsukohdassa voidaan hyödyntää nimettyjä parametreja haluttujen parametrien välittämiseksi

\begin{lstlisting}
type Luokka() =
    member this.Metodi(?ensimmainen, ?toinen) =
         # match valitsee Some-haaran jos muuttujalla on jokin arvo, muutoin valitaan None-haara         
         match ensimmainen with
                | Some arvo -> printfn "Ensimmäisen argumentin arvo %A" arvo
                | None -> printfn "Ei ensimmäistä argumenttia"
         match toinen with
                | Some arvo -> printfn "Toisen argumentin arvo %A" arvo
                | None -> printfn "Ei toista argumenttia"
         printfn "\n"        

// määritellään ohjelman aloituspiste.
[<EntryPoint>]
let main argv =

    let laskuri = Luokka()
    laskuri.Metodi()
    laskuri.Metodi(ensimmainen = 5) # haluttu argumentti voidaan valita nimetyn parametrin avulla
    laskuri.Metodi(toinen = 12)
    laskuri.Metodi(ensimmainen = 1, toinen = 1323)
    0
\end{lstlisting}

Tässä esimerkkitapauksessa tulostuu ensiksi siis “Ensimmäisen argumentin arvo 5” ja “Ei toista argumenttia” ensimmäisen funktiokutsun yhteydessä ja “Ensimmäisen argumentin arvo 1” ja “Toisen argumentin arvo 1323” toisen funktiokutsun yhteydessä.

F\#:n funktioparametrit ovat oletuksena immutaabeleja, minkä vuoksi se, välitetäänkö nämä parametrit arvona vai viitteenä ei ole mielekäs kysymys oletustapauksessa. Parametrit voidaan kuitenkin erikseen merkitä viiteparametreiksi byref-avainsanalla. Tällöin välitetyn parametrin täytyy olla myös muutettavissa, eli se on määritetty mutable-avainsanalla.
Esimerkiksi seuraava ohjelma

\begin{lstlisting}
let f(a:int byref)  =
    a <- a + 1

// määritellään ohjelman aloituspiste.
[<EntryPoint>]
let main argv =
    let mutable arvo = 4
    f(&arvo) # viiteparametri merkittävä, jolloin kutsukohdasta on selvää milloin muuttujan arvo saattaa muuttua funktiokutsun johdosta
    printfn "%A" arvo
    0  
\end{lstlisting}

tulostaa 5


\subsection{PHP}

\lstset{language=PHP,
	basicstyle=\ttfamily,
	breaklines=true,
	columns=fullflexible}

PHP-kielessä funktiot ovat ensisijainen väline aliohjelmien toteuttamiseen. Luokat tarjoavat mahdollisuuden funktioiden ja muuttujien kapselointiin. Parametrinvälitys on erityisen tärkeässä roolissa PHP:ssä, sillä funktiot ovat litteitä eikä niiden sisältä voi viitata ympäröivän leksikaalisen tai dynaamisen näkyvyysalueen muuttujiin (ks. artikkeli 2).

PHP:ssä käytetään parametrien ja paluuarvojen välityksessä oletusarvoisesti arvosemantiikkaa~\cite{man_php_args}. Funktio kopioi välitettävän parametrin arvon omaan paikalliseen muuttujaansa, eikä välitetyn parametrin arvo muutu vaikka funktion paikallista arvoa muutettaisiinkin.

Vaihtoehtoisesti parametrinvälityksessä voidaan käyttää viitesemantiikkaa~\cite{man_php_args}, esimerkiksi seuraavaan tapaan: 

\begin{lstlisting}
function lisaa_yhdella(&$luku) {
    $luku++;
}

$luku = 3;
lisaa_yhdella($luku);
echo "luku = " . $luku . "\n";
\end{lstlisting}

Edellinen ohjelma tulostaa suoritettaessa viestin: “luku = 4”.

Toisaalta myös paluuarvojen välityksessä voidaan käyttää viitesemantiikkaa~\cite{man_php_ret_ref}. Seuraava esimerkki demonstroi, miten funktion staattista muuttujaa voidaan muokata funktion ulkopuolelta, kun funktio palauttaa sen viitteen.

\begin{lstlisting}
function &test() {
    static $arr = array();
    array_push($arr, 'foo');
    return &$arr;
}

$arr = test();
array_push($arr, 'bar');
$arr = test();
print_r($arr);
\end{lstlisting}

Edellinen ohjelma tulostaa suoritettaessa:
\begin{lstlisting}
Array
(
    [0] => foo
    [1] => bar
    [2] => foo
)
\end{lstlisting}

mistä nähdään, että test-funktion paikallista staattista muuttujaa arr on tosiaankin päästy muokkaamaan funktion ulkopuolelta. On syytä huomata, että \&-merkki tulee laittaa sekä funktion nimen eteen, että palautettavan lausekkeen eteen, muutoin käytössä on arvosemantiikka.

PHP:n parametreille voidaan antaa oletusarvoja~\cite{man_php_args}. Oletusarvon omaavien muuttujien tulee sijaita funktion parametrilistan lopussa, siis oletusarvottomien parametrien oikealla puolella. Seuraava esimerkki demonstroi oletusarvojen käyttöä:

\begin{lstlisting}
function tulosta($nimi, $luku = 4, $taulukko = array("a", "b")) {
    $out = "Nimi: " . $nimi . " Luku: " . $luku . " Kirjaimet:";
    foreach($taulukko as $i) $out .= " " . $i;
    return $out;
}

echo tulosta("Jeppe", 8) . "\n";
\end{lstlisting}

Ohjelmaa suoritettaessa tulostuu: “Nimi: Jeppe Luku: 8 Kirjaimet: a b”.

\section{Virheenhallinta}

\subsection{F\#}

\lstset{
	language=FSharp,
	basicstyle=\ttfamily,
	breaklines=true,
	columns=fullflexible
}

F\#:ssa poikkeustyyppi~\cite{msn_exceptions} voidaan määrittää seuraavasti

\begin{lstlisting}
  exception Nimi of tyyppi
  exception Nimi of tyyppi * tyyppi * tyyppi # useita tyyppejä
\end{lstlisting}

ja poikkeus luoda komennolla
\begin{lstlisting}
  raise(Nimi(argumentit-poikkeuksen-haluamilla-tyypeillä))
\end{lstlisting}

esimerkiksi
\begin{lstlisting}
  exception OmaPoikkeus of string * int * float
 
  def func a =
    raise(OmaPoikkeus("Jotakin meni pieleen...", 10, 123.4))
\end{lstlisting}

F\#:ssa on hieman Javamainen try\with\finally-rakenne poikkeusten käsittelyyn~\cite{msn_try}. Suurena erona on kuitenkin se, että with ja finally -lohkot eivät voi kuulua samaan try-lohkoon yhtä aikaa. Jos kummallekin lohkolle on tarvetta, tarvitaan kaksi sisäkkäistä try-lohkoa, kuten seuraavassa esimerkissä

\begin{lstlisting}
exception OmaPoikkeus of string * int * float
exception ToinenPoikkeus of string

let f arvo  =
    if arvo > 10 then
        raise(OmaPoikkeus("tekstiä", 123, 123.4))
    else
        raise(ToinenPoikkeus("abcdef"))

// määritellään ohjelman aloituspiste.
[<EntryPoint>]
let main argv =

    try
            try
                    f(5)
            with   
                    | OmaPoikkeus(teksti, nro, floatNro) -> printfn "%A" (teksti, nro, floatNro)
                    | ToinenPoikkeus(str) -> printfn "%A" str
    finally
            printf "Vielä tänne"
    0
\end{lstlisting}

Tämä esimerkki tulostaa "abcdef" ja "Vielä tänne"


Kielessä on myös vielä failwith-avainsana~\cite{msn_failwith}, joka automaattisesti luo Failure-tyyppisen, stringin hyväksyvän poikkeuksen. Näin omaa poikkeustyyppiä ei välttämättä tarvitse määrittää. Esimerkiksi seuraava esimerkki

\begin{lstlisting}
let f arvo  =
    if arvo = 0 then
            failwith "Ei saa olla nolla"
    else
            printf "%A" "Ei ollut nolla"

// määritellään ohjelman aloituspiste.
[<EntryPoint>]
let main argv =
    try
            f(0)
    with   
            | Failure(str) -> printfn "%A" str
    0
\end{lstlisting}

tulostaa tekstin "Ei saa olla nolla"

    
\subsection{PHP}

\lstset{language=PHP,
	basicstyle=\ttfamily,
	breaklines=true,
	columns=fullflexible}

PHP:n poikkeuksien käsittely on samankaltaista kuin Javassa ja C++:ssa. Alirutiinit voidaan määritellä “heittämään” (throw) poikkeuksia, joita käsitellään try-catch-rakenteen avulla~\cite{man_php_exceptions}. Esimerkiksi:

\begin{lstlisting}
function jaa($jaettava, $jakaja) {
    if($jakaja == 0) {
        throw new Exception(‘Nollalla jakaminen ei onnistu’);
    }
    return $jaettava / $jakaja;
}

try {
    echo jaa(1, 2) . ‘\n’;    
} catch(Exception $e) {
    echo $e->getMessage() . ‘\n’;
} finally {
    echo “Ohjelma suoritettu.\n”;
}
\end{lstlisting}

Esimerkissä funktio jaa heittää poikkeuksen, jos parametri jakaja on arvoltaan 0. Try-catch rakenteessa poikkeus otetaan vastaan. Joka tapauksessa suoritetaan ohjelman finally-lohko, joka tulostaa merkkijonon “Ohjelma suoritettu.”

Ohjelmoijan on mahdollista tehdä omia poikkeuksia periytymisen avulla~\cite{man_php_extended_exceptions}. Käyttäjä voi luoda oman luokkansa, joka perii Exception-luokan ominaisuudet. Esimerkiksi class OmaPoikkeus extends Exception { }. Tällöin poikkeus heitetään funktiossa komennolla throw new OmaPoikkeus(‘virheviesti’).
