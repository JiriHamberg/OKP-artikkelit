 \lstset{%
        inputencoding=utf8,
            extendedchars=true,
            literate=%
            {å}{{\r{a}}}1
			{ä}{{\"a}}1
			{ö}{{\"o}}1
			{Å}{{\r{A}}}1
			{Ä}{{\"A}}1
			{Ö}{{\"O}}1
}

\lstset{
	language=FSharp,
	basicstyle=\ttfamily,
	breaklines=true,
	columns=fullflexible
}

\section{F\#}

\subsection{Tyypitys ja tyyppipäättely}

F\# on vahvasti ja staattisesti tyypitetty ohjelmointikieli~\cite{wiki_fs_programming}. F\#:ssa on käytössä tyyppipäättely, jonka ansiosta kääntäjä päättelee arvojen, muuttujien, parametrien ja paluuarvojen tyypit useimmissa tilanteissa automaattisesti, ellei ohjelmoija ole ilmaissut tyyppiä eksplisiittisesti~\cite{msn_type_inference}. Esimerkiksi seuraavanlaisen funktion määritelmän tapauksessa

\begin{lstlisting}
let f a b = a + b
\end{lstlisting}

kääntäjä päättelee f:lle seuraavanlaisen tyypin:

\begin{lstlisting}
f : a:int -> b:int -> int
\end{lstlisting}

mikä tarkoittaa, että f on funktio, joka ottaa parametreinaan kaksi kokonaislukua ja palauttaa kokonaisluvun. Tyyppipäättely ei rajoitu vain funktion omaan leksikaaliseen ympäristöön, vaan tyyppipäättelyn tulokseen vaikuttavat kaikki funktioon viittaavat määritelmät, kuten seuraavassa esimerkissä:

\begin{lstlisting}
let f a b = a + b
let g = f 3.0 2.0
\end{lstlisting}

Nyt kääntäjä tekee seuraavat tyyppipäätelmät:

\begin{lstlisting}
f : a:float -> b:float -> float
g : float
\end{lstlisting}

F\# kielen tyypitys on varsin vahva. Esimerkiksi numeerisille tyypeille ei suoriteta implisiittisiä tyyppimuunnoksia. Siksi seuraava johtaa käännösvirheeseen: 

\begin{lstlisting}
3.0 + 2
\end{lstlisting}

Kääntäjä tuottaa virheilmoituksen: “The type 'int' does not match the type 'float'”. Implisiittisten tyyppimuunnosten puuttuminen ja tyyppipäättely voivat yhdessä johtaa F\#-kieleen tottumattomalle ohjelmoijalle varsin erikoiselta vaikuttaviin virheisiin. Esimerkiksi seuraava ohjelma johtaa käännösvirheeseen:

\begin{lstlisting}
let f a b = a + b
let g = f 3.0 2.0
let h = f 3 2
\end{lstlisting}

Edellisessä esimerkissä kääntäjä päättelee g:n määritelmän perusteella, että funktion f: tyypin on oltava f : a:float -> b:float -> float, jolloin h:n määritelmä on virheellinen, sillä int-tyyppistä arvoa ei voida implisiittisesti muuntaa float-tyyppiseksi F\#:ssa.

Tyyppipäättely ei välttämättä toimi kaikissa tilanteissa. Tällöin ohjelmoija voi kirjoittaa arvoille tyyppiannotaatiota, jotka kääntäjä huomio tyyppipäättelyä suorittaessaan. Esimerkiksi seuraava määritelmä johtaa käännösvirheeseen:

\begin{lstlisting}
let f x = x.Length #käännösvirhe!
\end{lstlisting}

Lisäämällä parametrille x tyyppiannotaatio, päästään virheestä eroon:

\begin{lstlisting}
let f (x: string) = x.Length #f: string -> int
\end{lstlisting}


\subsection{Algebralliset ja rekursiiviset tyypit sekä tyyppiparametrit}

F\# tukee algebrallisia ja rekursiivisia tyyppejä~\cite{wiki_algebraic_data_type}. Yksinkertainen viikonpäivät koodaava summatyyppi voidaan määritellä seuraavasti:

\begin{lstlisting}
type viikonpaiva = Ma | Ti | Ke | To | Pe | La | Su
\end{lstlisting}

Edellinen määritelmä tarkoittaa, että viikonpaiva on tyyppi, jonka jäseniä ovat täsmälleen Ma, Ti, Ke, To, Pe, La ja Su.

Tulotyypit ovat rakenteisia tyyppejä, joiden arvojen joukko on muiden tyyppien arvojoukkojen Karteesinen tulo. Tulotyypin tekijätyypeiksi kelpaavat siis F\#:n alkeistyyppien (ks. artikkeli 1) ohella myös rakenteiset tyypit. Yksinkertainen merkkijonojen ja kokonaislukujen tulotyyppi voidaan määritellä esimerkiksi seuraavasti: 

\begin{lstlisting}
type henkilo = Henkilo of string * int
\end{lstlisting}

Edellinen määrittelee tyypin henkilo, jonka jäseniä ovat kaikki datakonstruktorilla Henkilo tuotetut string-int -parit. Algebrallisia tyyppejä käytetään tyypillisesti yhdessä hahmonsovituksen kanssa. Seuraava ohjelmaesimerkki havainnollistaa hahmonsovituksen periaatteita yllä määritellyn henkilo-tyypin tapauksessa:

\begin{lstlisting}
type henkilo = Henkilo of string * int
let nimi = function #nimi on tyyppiä henkilo -> string
  | Henkilo (a, _) -> a

[<EntryPoint>]
let main args =
  let aino = Henkilo ("Aino", 35)
  printfn "Hei %s" (nimi aino)
  0
\end{lstlisting}

Yllä oleva ohjelma tulostaa suoritettaessa viestin “Hei Aino”.

Summa- ja tulotyyppejä voidaan yhdistellä mielivaltaisiksi algebrallisiksi tyypeiksi. Algebralliset tyypit voivat F\#-kielessä olla myös rekursiivisia, eli algebrallisen tyypin T tulotyypit voivat sisältää jäseninään tyyppiä T. Seuraavassa esimerkissä on määritelty tyyppi puu, joka on kokonaislukuarvoinen binääripuu ja funktio puuSumma, joka laskee puu-tyyppisen parametrin lehtisolmujen summan.	

\begin{lstlisting}
type puu =
  | Lehti of int
  | Oksa of puu*puu

let rec puuSumma = function
  | Lehti (x) -> x
  | Oksa (p1, p2) -> puuSumma p1 + puuSumma p2

[<EntryPoint>]
let main args =
  let p = Oksa(
	Oksa(
  		Lehti(4),
  		Lehti(3)
	),
	Lehti(5)
  )
  printfn "puuSumma =  %d" (puuSumma p)
  0
\end{lstlisting}

Edellisen ohjelman suorittaminen tuottaa tulosteen “puuSumma =  12”. 
\par
F\# tukee myös geneerisiä tyyppejä tyyppiparametrien muodossa~\cite{msn_generics}. Seuraavassa esimerkissä edellisen esimerkin binääripuusta on tehty geneerinen versio. Myös puuSumma-funktiosta on tehty geneerinen versio.

\begin{lstlisting}
type puu<'a> =
  | Lehti of 'a
  | Oksa of puu<'a>*puu<'a>

let rec puuSumma<'a> (p: puu<'a>) (f: 'a -> int) = match p with
  | Lehti (x) -> f x
  | Oksa (p1, p2) -> puuSumma p1 f + puuSumma p2 f

[<EntryPoint>]
let main args =
  let p = Oksa(
	Oksa(
  	Lehti("aaaaaa"),
  	Lehti("bbb")
	),
	Lehti("c")
  )
  #lasketaan puun sisältämien merkkijonojen pituuksien summa
  printfn "puuSumma =  %d" (puuSumma p (fun (x: string) -> x.Length))
  0 #pääohjelman paluuarvo
\end{lstlisting}

Yllä olevassa ohjelmassa on määritelty geneerinen binääripuu-tyyppi “puu”, jolla on tyyppiparametri ‘a ja funktio puuSumma, joka ottaa parametreinaan tyyppiä ‘a sisältävän binääripuun ja funktion f, joka muuttaa ‘a-tyyppisen arvon kokonaisluvuksi. Pääohjelmassa on määritelty merkkijonoja sisältävä binääripuu, johon on sovellettu funktiota puuSumma puun sisältämien merkkijonojen pituuksien summan laskemiseksi. Ohjelman suorittaminen tuottaa tulosteen “puuSumma =  10”. Tyyppipäättely tekee tyyppiparametrien käytöstä varsin miellyttävää: pääohjelmassa ei tarvitse erikseen kertoa, että p on tyyppiä puu<string>, vaan tyyppiparametrin ‘a tyyppi päätellään automaattisesti. 
\par
Tyyppiekvivalenssi on F\#-kielessä läpinäkymätöntä: esimerkiksi kahden tulotyypin, joilla on samat tulontekijätyypit välillä ei vallitse tyyppiekvivalenssia. Tulotyypit erotetaan hahmontunnistuksessa toisistaan datakonstruktorien avulla.


\section{PHP}

\lstset{language=PHP,
	basicstyle=\ttfamily,
	breaklines=true,
	columns=fullflexible}

PHP on dynaamisesti tyypitetty kieli, joten muuttujille ei erikseen tarvitse antaa tyyppiä ja muuttujan sisältämän arvon tyyppi voi muuttua ajonaikaisesti. Esimerkiksi seuraava koodi~\cite{man_php_types}

\begin{lstlisting}
$a = 5;
echo gettype($a);

$a = "Terve";
echo gettype($a);
\end{lstlisting}

tulostaa “integer” ja “string”.
\par
Arvoilla kuitenkin on tyyppi kuten yllä olevasta koodista nähdään. PHP tukee neljää skalaarityyppiä: kokonaislukuja (integer), liukulukuja (float), totuusarvoja (boolean) ja merkkijonoja (string). Näiden lisäksi tukee taulukoita ja olioita (array, object) rakenteisina tyyppeinä. Kieli sisältää vielä kaksi erikoistyyppiä, null ja resource. Null-tyypin ainoa arvo on NULL, joka ilmaisee arvon puuttumisen. Arvo voi olla vain sellaisilla muuttujilla, joille ei ole asetettu arvoa tai joiden arvo on poistettu unset-funktiolla tai sijoittamalla muuttujaan NULL. Resource-tyyppi puolestaan on erikoistyyppi kielen ulkopuolisille resursseille kuten tiedostokahvoille tai tietokanta- ja verkkoyhteyksille.
\par
PHP sisältää runsaasti implisiittisiä muunnoksia, joten se on tyypitykseltään heikko. Esimerkiksi seuraava koodinpätkä

\begin{lstlisting}
$a = "25";
$b = 20;
$c = $a + $b;

echo $c;
\end{lstlisting}

tulostaa 45, koska + -operaattori pyrkii muuntamaan operandinsa kokonaisluvuiksi. Sivuhuomiona tässä voi mainita, että katenaatio-operaatio ilmaistaan kielessä poikkeuksellisesti pisteellä: Jos ylläoleva koodi olisikin seuraava

\begin{lstlisting}
$a = "25";
$b = 20;
$c = $a . $b;

echo $c;
\end{lstlisting}

olisi tulostunut 2520.
\par
Implisiittiset konversiot aiheuttavat joskus hyvin vaikeasti löydettäviä virheitä. Esimerkiksi seuraava koodinpätkä

\begin{lstlisting}
$a = 0;
$b = "testi123";

echo var_dump($a == $b);
\end{lstlisting}

tulostaa boolean(true), eli ehto toteutuu. Tämä johtuu siitä, että PHP muuntaa muuttujan \$b merkkijonoksi koska \$a on merkkijono. Jos merkkijonoa ei voi muuntaa numeroksi, se palauttaa arvon nolla, kuten nyt käy koska “testi123” ei ala luvulla. Näin vertailu on tosi. Toisaalta myös seuraavassa koodinpätkässä

\begin{lstlisting}
$a = 1234;;
$b = "1234ABCDEF";

echo var_dump($a == $b);
\end{lstlisting}

ehto on tosi: koska \$b alkaa luvulla, muunnetaan se kokonaisluvuksi 1234 ja vertailu on jälleen tosi. 
\par
Vertailuoperaattoreiden kohdalla tyyppimuunnokset ovat joskus myös odottamattomia. Esimerkiksi seuraavassa koodissa~\cite{man_php_typejuggling}

\begin{lstlisting}
echo var_dump("123" < "456A");
echo var_dump("456A" < "78");
echo var_dump("78" < "123");
\end{lstlisting}

jokainen ehto on tosi. Tämän vuoksi seuraavassa koodissa~\cite{reddit_example}

\begin{lstlisting}
$a = array("123", "456A", "78"); sort($a); print_r($a);
$b = array("456A", "78", "123"); sort($b); print_r($b);
\end{lstlisting}

ensimmäisen taulukon järjestys on 123, 456A ja 78 kun taas toisen taulukon järjestys on 456A, 78, 123. PHP:n tyyppimuunnosten kanssa on siis oltava todella tarkkana.



%http://en.wikipedia.org/wiki/F_Sharp_%28programming_language%29
%https://msdn.microsoft.com/en-us/library/dd233180.aspx
%http://en.wikipedia.org/wiki/Algebraic_data_type
%https://msdn.microsoft.com/en-us/library/dd233215.aspx
%http://php.net/manual/en/language.types.intro.php 
%http://php.net/manual/en/language.types.type-juggling.php
%http://www.reddit.com/r/lolphp/comments/2fxgu7/the_codeless_code_case_161_triangle_the_abbey_of/cke7cla

