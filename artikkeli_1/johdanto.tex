\section{Johdanto}
F\# on Microsoft Researchin kehittämä moniparadigmakieli, jonka ensimmäinen versio julkaistiin 2005. Se mahdollistaa sekä funktionaalisen että imperatiivisen ohjelmoinnin~\cite{wiki_fs_programming}. Kieli pohjautuu ML-kieleen jopa niin läheisesti, että sitä voidaan pitää kielen varianttina~\cite{wiki_ml}. F\# on saanut vaikutteita myös C\#:sta ja OCamlista. Kieli on alusta asti suunniteltu Microsoftin CLR-virtuaalikoneella suoritettavaksi, jolloin siinä on mahdollista käyttää myös esimerkiksi C\#-kielellä kirjoitettuja kirjastoja.
\par
PHP on imperatiivinen, proseduraalinen ja oliopohjainen dynaamisesti tyypitetty kieli. Sen ensimmäinen versio julkaistiin 2005. Kieli on saanut runsaasti vaikutteita C\textbackslash C++:sta, Perlistä ja Javasta. Kielen alkuperäinen kehittäjä on Rasmus Lerdorf, joka kehitti kielen alunperin henkilökohtaisten kotisivujensa tarpeisiin. Kieli on näin alusta alkaen keskittynyt web-palvelinohjelmointiin ja tämä on edelleen sen suosituin käyttökohde. PHP kärsii kuitenkin näistä lähtökohdista jonkin verran, sillä sen alkuvuosina siihen lisättiin ominaisuuksia ilman tarkkaa suunnittelua. Tämä on johtanut muun muassa epäjohdonmukaiseen nimeämiskäytäntöön ja funktioiden argumenttijärjestykseen standardikirjastossa.