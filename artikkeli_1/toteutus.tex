\section{Toteutus}
PHP:n toteutukset ovat lähes poikkeuksetta tulkattuja. Referenssitoteutus  käyttää Zend Engine-nimistä tulkkia, joka kääntää PHP-ohjelman sisäiselle välikielelle, jota tulkki suorittaa. Koska PHP-kielellä ei ole virallista määrittelyä, sen referenssitoteutus määrittää kielen semantiikan. Näin Zend Enginen oma toteutus määrittää pitkälti kielen semantiikan. Tämän vuoksi kilpailevien toteutuksien on vaikeaa saavuttaa täydellistä yhteensopivuutta, koska niiden tulisi kopioida Zend Enginen toteutus täydellisesti~\cite{wiki_php}.

Toinen tärkeä toteutus PHP:lle on Facebookin HipHop Virtual Machine (HHVM). Myös tämä toteutus kääntää aluksi ohjelman tavukoodiksi, jota suoritetaan. Tämän lisäksi tavukoodia voidaan kääntää konekielelle ajonaikaisesti (Just-In-Time-käännös), minkä ansiosta HHVM on jopa kuusi kertaa nopeampi kuin referenssitoteutus \cite{wiki_hiphop}. HHVM pohjautuu aikaisempaan toteutukseen, jossa PHP käännettiin ensiksi C++-kielelle, josta se käännettiin konekielelle. Tämä aiempi toteutus ei kuitenkaan tukenut kaikki PHP:n ominaisuuksia, kuten eval()-funktiota~\cite{wiki_hiphop}.

F\#-kieltä puolestaan yleensä suoritetaan CLR-virtuaalikonella (Common Language Runtime), joka on Microsoftin vastine Javan virtuaalikoneelle. Käännösvaiheessa F\#-kieli käännetään CLR-tavukoodiksi. Tätä tavukoodia voidaan joko suorittaa suoraan tulkilla, tai kääntää ajonaikaisesti konekielelle. F\#-kielelle on myös kääntäjiä, jotka tuottavat Javascriptiä~\cite{wiki_fs_programming}.