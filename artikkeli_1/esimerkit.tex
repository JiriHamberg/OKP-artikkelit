\section{Esimerkkejä}

 \lstset{%
        inputencoding=utf8,
            extendedchars=true,
            literate=%
            {å}{{\r{a}}}1
			{ä}{{\"a}}1
			{ö}{{\"o}}1
			{Å}{{\r{A}}}1
			{Ä}{{\"A}}1
			{Ö}{{\"O}}1
}

\lstset{
	language=FSharp,
	basicstyle=\ttfamily,
	breaklines=true,
	columns=fullflexible
}


\lstinputlisting{artikkeli_1/fibo.fs}
\par
Esimerkkiohjelma PHP:llä. Ensimmäinen esimerkki on rekursiivinen toteutus funktiosta, joka hakee $\$min\_value$-parametri:a suuremman fibonaccin luvun. 

\lstset{language=PHP}
\begin{lstlisting}
function fibonacci($first, $second, $min_value) {
    if($first + $second <= $min_value) {
        return fibonacci($second, ($first + $second), $min_value);
    }
    return $first + $second;
}
\end{lstlisting}

Funktiolle annetaan kolme parametria, joista kaksi ensimmäistä ovat fibonaccin lukujonoon kuuluvia lukuja. Funktiota kutsutaan esimerkiksi seuraavasti:
\[ \$next_fib = fibonacci(0, 1, 5); \]
Muuttujan $\$next_fib$ arvoksi tulee tässä tapauksessa 8.
Funktiosta voi myös tehdä iteratiivisen version, joka vastaanottaa vain yhden parametrin.

\begin{lstlisting}
function fibonacci_iterative($min_value) {
  $first = 0;
  $second = 1;
  $now = 0;
  while($now <= $min_value) {
     $now = $first + $second;
     $first = $second;
     $second = $now;
  }
  return $now;
}
\end{lstlisting}
